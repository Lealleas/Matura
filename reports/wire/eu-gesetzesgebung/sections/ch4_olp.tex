\section{Ordinary legislative procedure}
\label{sec:OLP}

\subsection{Step 1: Legislative proposal}

The procedure is launched when the \textbf{European Commission} submits a legislative proposal to the Council and the European Parliament. At the same time it sends the proposal to national parliaments and, in some cases, the Committee of the Regions and the Economic and Social Committee for their examination.
\\
Legislative proposals are adopted by the college of the members of the Commission either by written procedure (i.e. the text is not discussed) or by oral procedure (with a discussion). If a vote is requested, the Commission decides by simple majority.\nextline

\textbf{Right of initiative}

The \textbf{Commission} is the only EU institution empowered to initiate EU legal acts. It submits proposals for EU legal acts on its own initiative, at the request of other EU institutions or following a citizens' initiative.

The \textbf{Council} (by a simple majority of its members) may request the Commission to carry out studies and submit any appropriate legislative proposals.

The \textbf{Parliament} (by a majority of its component members) may also ask the Commission to submit legislative proposals.

In specific cases, defined in the treaties, the ordinary legislative procedure can be launched:
\begin{itemize}
\item on the initiative of \textbf{a quarter of the member states}(when the proposal concerns judicial cooperation in criminal matters or police cooperation)
\item on a recommendation from the \textbf{European Central Bank} (on proposals concerning the statute of the European system of central banks and of the European Central Bank)
\item at the request of the \textbf{Court of Justice of the EU} (on matters relating to the statute of the Court, establishment of specialised courts attached to the General Court, etc.)
\item at the request of the \textbf{European Investment Bank}
\end{itemize}

\subsection{Step 2: First reading}
The \textbf{European Parliament} examines the Commission's proposal and may:
\begin{itemize}
\item adopt it or
\item introduce amendments to it
\end{itemize}
\clearpage
After that the \textbf{Council} may:
\begin{itemize}
\item decide to accept the Parliament's position: in such a case the legislative act is adopted
\item amend the Parliament's position: the proposal is returned to the Parliament for a second reading
\end{itemize}

\textbf{Deadline}: There is no time limit on the first reading at the Parliament and at the Council

\textbf{Resulting documents:}
\begin{itemize}
\item \textbf{Legislative act - regulation} (binding in its entirety and directly applicable in all member states), directive (binding, in terms of the result to be achieved, may be addressed to all or only some member states, and member states are free to chose the form and methods to implement the directive) or decision (binding in its entirety for those to whom it is addressed) of the Parliament and of the Council
\item \textbf{Council's position} at the first reading, if the Council decides to amend the Parliament's position
\item \textbf{General approach} - a political agreement at Council level that may be adopted pending first reading position of the Parliament 
\end{itemize}

\textbf{Intermediate steps prior to a position at first reading}

Before the European Parliament delivers its opinion, the Council may adopt a \textit{general approach}. The Council uses this document to give the Parliament an idea of its position on the Commission's legislative proposal. A general approach can speed up the legislative procedure and make it easier to reach an agreement between the Parliament and the Council. 

The Council, the Parliament and the Commission can also organise informal interinstitutional meetings, known as \textit{trilogues}, to help them reach an agreement. These meetings are attended by representatives from the Parliament, the Council and the Commission. 

There is no set rule regarding the content of trilogues, so they may vary from technical discussions to political discussions involving ministers and commissioners. Trilogues can also be used to reach an agreement on legislative amendments between the Parliament and the Council. However, the resulting agreement is informal and has to be approved according to the rules of procedure of each of the institutions.
\clearpage

\subsection{Step 3: Second reading}
The \textbf{European Parliament} examines the Council's position and either:
\begin{itemize}
	\item approves it: the act is adopted
	\item rejects it: the act will not enter into force and the whole procedure ends
	\item proposes amendments and returns the proposal to Council for a second reading
\end{itemize}

The \textbf{Council} examines the Parliament's second reading position and either:
\begin{itemize}
	\item approves all of the Parliament's amendments: the act is adopted
	\item does not approve all amendments: the conciliation committee is convened
\end{itemize}

The Council votes \textbf{by qualified majority} on the Parliament's amendments for which the Commission has delivered a positive opinion. It \textbf{votes unanimously} on the Parliament's amendments for which the Commission has delivered a negative opinion. The Council can only react to the Parliament's amendments.

\textbf{Deadline}: Three months for each institution, with a possible extension of one month.

\textbf{Resulting documents:}
\begin{itemize}
	\item \textbf{European Parliament legislative resolution} on the Council position at first reading, if the Parliament approves the Council's position (in such cases, the legislative act is adopted and published as a directive, a regulation or a decision of the Parliament and of the Council) 
	\item \textbf{Position of the European Parliament} adopted at second reading, if the European Parliament votes to introduce changes to the Council's position
	\item If the Council approves the Parliament's second reading position, the legislative act is adopted. It is subsequently published as a \textbf{directive, a regulation or a decision} of the Parliament and of the Council
	\item If the Council does not approve the Parliament's second reading position, there is \textbf{no official document}
\end{itemize}

\clearpage
\subsection{Step 4: Conciliation}
A \textbf{conciliation committee} is convened if the Council does not approve all of the Parliament's amendments at the second reading. It is composed of an equal number of members of the Parliament and Council representatives. It has to agree on a text that would be acceptable to both institutions. \\

If the committee:
\begin{itemize}
	\item does not agree on a joint text, the legislative act is not adopted and the procedure is ended
	\item agrees the joint text, that text is forwarded to the Parliament and the Council for a third reading
\end{itemize}

\textbf{Voting in the conciliation committee:}

The Parliament delegation to the conciliation committee approves the joint text by an absolute majority of votes.
\\

The Council representatives generally vote by qualified majority (in some cases unanimity is required).
\\

\textbf{Deadline}: The conciliation committee must be convened within 6 weeks, with a possible extension to 8 weeks. The committee then has 6 weeks to agree on a joint text.

\textbf{Resulting documents:}
\begin{itemize}
\item Joint text approved by the conciliation committee, if an agreement is reached. The wording of the joined text cannot be changed by the two institutions.
\end{itemize}

\clearpage
\subsection{Step 5: Third reading}

The \textbf{European Parliament} examines the joint text. It may:
\begin{itemize}
\item reject it or fail to act on it: the proposal is not adopted and the procedure ends
\item approve the text: if the Council does the same, the legislative act is adopted
\end{itemize}

The Parliament approves the joint text by a simple majority of votes cast. \\

The \textbf{Council} examines the joint text. If it
\begin{itemize}
\item rejects it or does not act on it, the proposal will not enter into force and the procedure is ended
\item approves the text, and so does the Parliament, the legislative act is adopted
\end{itemize}

The Council approves the joint text by qualified majority. \\

\textbf{Deadline}: Both the Parliament and the Council must act within 6 weeks, starting from the date on which the joint text was approved.

\textbf{Resulting documents:}
\begin{itemize}
	\item The Parliament adopts a \textbf{legislative resolution }on the joint text approved by the conciliation committee in which it either approves or rejects the joint text.
	\item If the joint text is approved by both institutions, the legislation is published as a \textbf{directive, a regulation or a decision} of the Parliament and of the Council.
\end{itemize}

\textbf{Rejection of the proposal}

If a legislative proposal is rejected at any stage of the procedure, or the Parliament and the Council cannot reach a compromise, the proposal is not adopted and the procedure ends. \\

A new procedure can start only with a new proposal from the Commission.

\subsection{Emergency break procedure}
By means of this procedure a member state can request the Council to suspend the legislative process. The proposal is then put to the European Council. \\
The emergency break procedure can only be invoked for fundamental proposals in a limited number of policy areas (foreign and defence policy, social security, judicial cooperation in criminal matters).

\subsection{Exceptions}
In a limited number of policy areas the Commission may submit proposals together with the member states, the European Court of Justice or the European Central Bank. In such cases the procedure is slightly modified. Council and Parliament now have to inform the Commission of their views at each step of the legislative process, and they can ask the Commission for an opinion at each step as well. The Commission can issue an opinion at any given moment in the legislative proces at their own initiative. \\

Voting procedures, the right of amendment and the rules on adopting or rejecting a proposal remain the same. \\

The policy areas this exception can be applied to are:
\begin{itemize}
	\item closer co-operation in criminal matters
	\item administrative co-operation in security aspects of home affairs
	\item measures regarding the usage of th euro
	\item establishing the statute of the European Central Bank
	\item establishing the statutes of the Court of Justice and establishing specialised courts
\end{itemize}
\clearpage





























