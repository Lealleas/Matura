\section{Potential implications of reaching and not raising the debt limit}
\label{sec:Implications}

If the federal government were to reach the debt limit and Treasury were to exhaust its alternative strategies for remaining under the debt limit, then the federal government would need to rely solely on incoming revenues to finance obligations. If this occurred during a period when the federal government was running a deficit, the dollar amount of newly incurred federal obligations would continually exceed the dollar amount of newly incoming revenues.
\nextline
It is not possible for CRS to specifically predict what Congress, the President, the Office of Management and Budget (OMB), Treasury, federal agencies, and financial markets would do in certain situations. Nevertheless, it is possible to scope out some aspects of what could happen under a specific scenario, in which the federal government is no longer able to issue debt, has exhausted alternative sources of cash, and therefore is dependent upon incoming receipts or other sources of funds to provide any cash that is necessary to pay federal obligations. That said, CRS cannot state the full range of events that may occur if the described scenario were to actually take place.
\nextline
In this scenario, the federal government implicitly would be required to use some sort of decision- making rule about whether to pay obligations in the order they are received, or, alternatively, to prioritize which obligations to pay, while other obligations would go into an unpaid queue. In other words, the federal government’s inability to borrow or use other means of financing implies that payment of some or all bills or obligations would be delayed.

\subsection{Possible Options for the Treasury}
Some have argued that prioritization of payments can be used by Treasury to avoid a default on selected federal obligations by paying interest on outstanding debt before other obligations. Treasury officials have maintained that the department lacks formal legal authority to establish priorities to pay obligations, asserting, in effect, that each law obligating funds and authorizing expenditures stands on an equal footing. In other words, Treasury would have to make payments on obligations as they come due.
\nextline
In contrast to this view, GAO wrote to then-Chairman Bob Packwood of the Senate Finance Committee in 1985 that it was aware of no requirement that Treasury must pay outstanding obligations in the order in which they are received. GAO concluded that “Treasury is free to liquidate obligations in any order it finds will best serve the interests of the United States.” In any case, if Treasury were to prioritize, it is not clear what the priorities might be among the different types of spending.
\nextline
While the positions of Treasury and GAO may appear at first glance to differ, closer analysis suggests that they merely offer two different interpretations of silence in statute with respect to a prioritization system for paying obligations. On one hand, GAO’s 1985 opinion posits that silence in statute with regard to prioritization simply leaves the determination of payment prioritization to the discretion of the Treasury Department. Conversely, Treasury appears to assert that the lack of specific statutory direction operates as a legal barrier, effectively preventing it from establishing a prioritization system.
\nextline
Subsequently, Treasury noted in 2011 that an attempt to prioritize payments was “unworkable” because adopting a policy that would require certain types of payments taking precedence over other U.S. legal obligations would merely be “a failure by the U.S. to stand behind its commitments.” In an August 2012 letter, the Treasury Inspector General also addressed this topic by stating, “Treasury officials determined that there is no fair or sensible way to pick and choose among the many bills that come due every day. Furthermore, because Congress has never provided guidance to the contrary, Treasury’s systems are designed to make each payment in the order it comes due.” At a hearing before the Senate Finance Committee in October 2013, Treasury Secretary Lew stated the following:

\begin{displayquote}
We write roughly 80 million checks a month. The systems are automated to pay because for 224 years, the policy of Congress and every president has been we pay our bills. You cannot go into those systems and easily make them pay some things and not other things. They weren't designed that way because it was never the policy of this government to be in the position that we would have to be in if we couldn't pay all our bills.\footnote{U.S. Congress, Hearing of the Senate Committee on Finance, The Debt Limit, 113th Congress, 1st Session, October 10, 2013. Transcript available on CQ.com at \url{http://www.cq.com/doc/congressionaltranscripts-4359941}.}
\end{displayquote}

\subsection{Possible Options for OMB}
It also is possible that OMB may use statutory authority to apportion or reapportion budget authority (i.e., the authority to incur obligations) that Congress has granted in appropriations, contract, and borrowing authority to delay expenditures and effectively establish priorities for liquidating obligations. OMB is required by statute to “apportion” these funds (e.g., quarterly) to prevent agencies from spending at a rate that would exhaust their appropriations before the end of the fiscal year. If OMB were to use statutory apportionment authority to affect the rate of federal spending, its ability to do so would be constrained by the Impoundment Control Act of 1974, as amended. As noted earlier, the Impoundment Control Act does not prohibit the President from withholding funds, but establishes procedures for the President to submit formal requests to Congress either to defer (i.e., delay) spending until later or to rescind (i.e., cancel) the budget authority that Congress previously had granted.Although the use of OMB’s apportionment authority in the event of a debt limit crisis might delay the need to pay some obligations, use of the authority would not prevent obligations from remaining unpaid.

\subsection{Potential Impacts on Government Operations}
If the debt limit is reached and not increased, federal spending would be affected. Under normal circumstances, Treasury has sufficient financial resources to liquidate all obligations arising from discretionary and mandatory (direct) spending, the latter of which includes interest payments on the debt. If a lapse in raising the debt limit should prevent Treasury from being able to liquidate all obligations on time, it is not clear whether the distinction between different types of spending would be significant or whether the need to establish priorities would disproportionately impact one type of spending or another. It is also not clear whether the distinctions among different types of obligations, such as contract, grant, benefit, and interest payments, would prove to be significant.

\nextline
\textbf{Potential Impacts on Programs Generally}
A government that delays paying its obligations in effect borrows from vendors, contractors, beneficiaries, other governments, or employees who are not paid on time. Moreover, a backlog of unpaid bills would continue to grow until the government collects more revenues or other sources of cash than its outlays. In some cases, delaying federal payments incurs interest penalties under some statutes such as the Prompt Payment Act, which directs the government to pay interest penalties to contractors if it does not pay them by the required payment date, and the Internal Revenue Code, which requires the government to pay interest penalties if tax refunds are delayed beyond a certain date. The specific impacts of delayed payment would depend upon the nature of the federal program or activity for which funds are to be paid.

\nextline
\textbf{Potential Impacts on Programs with Trust Funds}
If Treasury delays investing a federal trust fund’s revenues in government securities, or redeems prematurely a federal trust fund’s holdings of government securities, the result would be a loss of interest to the affected trust fund. This could potentially worsen the financial situation of the affected trust fund(s) and accelerate insolvency dates.As noted earlier, Congress passed P.L. 104-121 to prevent federal officials from using the Social Security and Medicare Trust Funds for debt management purposes, except when necessary to provide for the payment of benefits and administrative expenses of the programs. Under P.L. 99-509, Treasury is permitted to delay investment in the TSP’s G-Fund and the Civil Service Retirement and Disability Trust Fund, and also to redeem prematurely assets of the Civil Service Retirement and Disability Trust Fund. However, the law also requires Treasury to make these funds whole after a debt limit impasse is resolved. The government maintains a number of other trust funds whose finances could potentially be harmed by delayed investment or early redemption in the absence of similar actions to make the trust funds whole after a debt limit impasse has ended.

\subsection{Potential Economic and financial Effects}
In addition to the potential impact on federal programs and activities if the debt limit is not increased, there may also be economic and financial consequences. A 1979 GAO report described the consequences of failing to increase the debt ceiling. GAO said the government had never defaulted on any of its securities, because cash has been available to pay interest and redeem them upon maturity or demand.60 Further, GAO said a default on the securities could have adverse effects on the economy, the public welfare, and the government’s ability to market future securities.

\begin{displayquote}
It is difficult to perceive all the adverse effects that a government default for even a short time would have on the economy and the public welfare. It is generally recognized that a default would preclude the government from honoring all of its obligations to pay for such things as employees’ salaries and wages; social security benefits, civil service retirement, and other benefits from trust funds; contractual services and supplies, and maturing securities.... At a minimum, however, the government could be subject to additional claims for interest on unredeemed matured debt and to claims for damages resulting from failure to make payments. But even beyond that, the full faith and credit of the U.S. government would be threatened. Domestic money markets, in which government securities play a major role, could be affected substantially.
\end{displayquote}

If the debt limit were reached and interest payments on debt were paid, it is not clear what the repercussions would be on the financial markets or the economy. If Treasury had to rely on incoming cash to pay its obligations, a significant portion of government spending would go unpaid. Removing a portion of government spending from the economy would leave behind significant economic effects and would have an effect on gross domestic product (GDP) by definition, all other things being equal.Further, if the government fails to make timely payments to individuals, service providers, and other organizations, these persons and entities would also be affected. Even if the government continued paying interest, it is not clear whether creditors would retain or lose faith in the government’s willingness to pay its obligations. If creditors lost this confidence, the federal government’s interest costs would likely increase substantially and there would likely be broader disruptions to financial markets.
\nextline
U.S. credit rating, an increase in federal and private borrowing costs, damage to the economic recovery, and broader disruptions to the financial system. Finally, the committee also warned that a prolonged delay in raising the debt limit could have negative consequences on the market before the time when default would actually occur.
