\section{A brief history of the federal debt limit}
\label{sec:History}

\subsection{Origins of the federal debt limit}
Congress has always placed restrictions on federal debt. Limitations on federal debt have helped Congress assert its constitutional powers of the purse, of taxation, and the initiation of war. Between World War I and World War II the form of statutory restrictions on federal debt evolved into an aggregate limit that applied to nearly all federal debt outstanding.
\newline \newline
Before World War I, Congress often authorized borrowing for specified purposes, such as the construction of the Panama Canal.31 Congress also often specified which types of financial instruments Treasury could employ, and specified or limited interest rates, maturities, and details of when bonds could be redeemed. In other cases, especially in time of war, Congress provided the Treasury with discretion, subject to broad limits, to choose debt instruments.32 Some opponents raised concerns that granting the Treasury Secretary authority to issue debt could affect monetary policies, which might tighten credit conditions. Proponents contended that federal borrowing would not disrupt settlements on such monetary issues reached in 1878 and 1890. Such concerns became moot after the establishment of the Federal Reserve System in 1913.
\newline \newline
For example, the War Revenue Act of 1898 allowed Treasury to use certificates of indebtedness, which had maturities of a year or less, and were used for short-term borrowing and cash management, as well as long-term bonds.33 For example, the 1898 War Revenue Act (30 Stat. 448-470) that funded Spanish-American War costs granted the Treasury Secretary the authority to have \$100 million outstanding in certificates of indebtedness with maturities under a year, which were mainly sold to large investors, banks, and other financial institutions. The act also allowed the Treasury to issue \$400 million in longer-term notes and bonds, which were made available to public subscription, allowing smaller investors to participate. Proponents of the act, however, made clear their intention to allow the Treasury Secretary substantial administrative leeway within those limits.
\newline \newline
Over time, the leeway granted the Treasury Secretary tended to expand. For example, the Second Liberty Bond Act of 1917, which helped finance the United States’ entry into World War I, dropped certain limits on the maturity and redemption of bonds.35 The act also incorporated unused borrowing capacity authorized by the First Liberty Bond Act (40 Stat 35; P.L. 65-3) and other previous borrowing acts.36 Separate limits for previous debt issues, however, were retained in the text of that act—an overall aggregated debt limit evolved later. Features of debt authorized by previous acts, such as the broad tax exemption for First Liberty Bond Act securities, remained intact.
\newline \newline
Subsequent borrowing measures were drafted as amendments to Second Liberty Bond Act until 1982.37 Setting debt policy by amendments to the Second Liberty Bond Act of 1917 rather than through original statutes reflected changes in legislative drafting practices at that time.
\newline \newline
In the 1920s, Congress provided Treasury Secretary Andrew Mellon with additional leeway in order to replace expensive older federal debt with cheaper new issues. Congress allowed Treasury to issue notes, a financial instrument issued extensively in the Civil War and rarely thereafter, and limited the amount of notes outstanding, rather than the sum of issuances, which gave greater Treasury flexibility to roll over debt. Savings certificates designed for small investors were also reintroduced.
\newline \newline
In the 1930s, Congress moved towards aggregate constraints on federal borrowing that allowed the Treasury greater ability to respond to changing conditions and more flexibility in financial management. In 1939, Congress eliminated separate limits on bonds and on other types of debt, which created the first aggregate limit (\$45 billion) that covered nearly all public debt.40 This measure gave the Treasury freer rein to manage the federal debt as it saw fit. Thus, the Treasury could issue debt instruments with maturities that would reduce interest costs and minimize financial risks stemming from future interest rate changes.41 While a separate \$4 billion limit for “National Defense” series securities was introduced in 1940, legislation in 1941 folded that borrowing authority back under an increased aggregate limit of \$65 billion.
\newline \newline
Although the Treasury was delegated greater independence of action on the eve of the United States’ entry into World War II, the debt limit at the time was much closer to total federal debt than it had been at the end of World War I. For example, the 1919 Victory Liberty Bond Act (P.L. 65-328) raised the maximum allowable federal debt to \$43 billion, far above the \$25.5 billion in total federal debt at the end of FY1919.43 By contrast, the debt limit in 1939 was \$45 billion, only about 10\% above the \$40.4 billion total federal debt of that time.

\subsection{World War II and after}
The debt ceiling was raised to accommodate accumulating costs for World War II in each year from 1941 through 1945, when it was set at \$300 billion. After World War II ended, the debt limit was reduced to \$275 billion. Because the Korean War was mostly financed by higher taxes rather than by increased debt, the limit remained at \$275 billion until 1954. After 1954, the debt limit was reduced twice and increased seven times, until March 1962 when it again reached \$300 billion, its level at the end of World War II. Since March 1962, Congress has enacted 75 separate measures that have altered the limit on federal debt. Most of these changes in the debt limit were, measured in percentage terms, small in comparison to changes adopted in wartime or during the Great Depression. Some recent increases in the debt limit, however, were large in dollar terms. For instance, in May 2003, the debt limit increased by \$984 billion and in February 2010 the debt limit was increased by \$1.9 trillion.




