\section{Concluding Comments}
\label{sec:Conclusio}

Since the late 1950s, the federal government increased its borrowing from the public in all years, except in 1969 following imposition of a war surcharge and in the period 1997-2001. The persistence of federal budget deficits has required the government to issue more and more debt to the public. The accumulation of Social Security and other trust funds, particularly after 1983 when recommendations of the Greenspan Commission were implemented, led to sustained growth in government-held debt subject to limit. The growth in federal debt held by the public and in intergovernmental accounts, such as trust funds, has periodically obliged Congress to raise the debt limit.
\nextline
Between August 1997, when the debt limit was raised to \$5,950 billion, and the beginning of 2002 in October 2001, federal budget surpluses reduced debt held by the public. From the end of 2001, the last fiscal year with a surplus, until the end of 2008, debt held by the public subject to limit grew by \$2,484 billion. Federal debt held in intergovernmental accounts grew steadily throughout the period, rising by \$1,743 billion since the beginning of 2002.
\nextline
In early 2001, the 10-year budget forecasts projected large and growing surpluses, indicating rapid reduction in debt held by the public. Some experts expressed concern about consequences of retiring all federal debt held by the public. Most long-term forecasts computed at that time, however, showed large deficits emerging once the baby boomers began to retire. Short-term forecasts projected continuous growth in debt held by government accounts, largely due to the difference between Social Security tax revenues and benefit payments. The combination of falling levels of publicly held debt and rising levels of debt held by government accounts moderated the expected growth of total debt. The moderate growth in total debt those projections had forecast was expected to postpone the need to increase the debt limit until late into the decade, when accumulating debt in government accounts would overtake reductions in debt held by the public.
\nextline
New budget projections released in early 2002 smashed expectations of large, persistent surpluses, and hopes for reductions in debt held by the public collapsed. The return to large federal deficits accelerated the growth of total debt. Increases in the debt limit would be necessary much sooner than previously expected.
\nextline
Early in 2003, the 2003 deficit and the persistent rise in debt held by government accounts drove the federal debt up to the \$6,400 billion limit in effect at the time. The Treasury avoided breaching the limit into May. Congress adopted a debt limit increase of \$934 billion on May 23, 2003, that provided enough room for the growing federal debt through the fall of 2004. The debt limit increase passed by Congress late in 2004 was expected, at the time, to accommodate the government’s debt growth well into 2005, if not into early 2006. In late December 2005, and early in 2006, the Treasury informed Congress that the limit would be reached between mid- February and mid-March 2006. On March 16, 2006, the Senate passed the House-initiated debt limit increase, raising the debt limit to \$8,965 billion. The debt limit crisis was resolved when the President signed the debt limit increase on March 20.
\nextline
Smaller than expected deficits in 2006 and 2007 postponed, but did not end the need for a new, higher debt limit. The House passed legislation in May 2007 to raise the debt limit. The Senate passed the measure on September 27, which the President signed on September 29. Turmoil in some financial markets in August 2007, according to some observers, appeared to constrain contention over the debt limit increase.
\nextline
The 2008 economic slowdown, which reduced federal tax revenues and increased federal outlays, caused federal deficit spending to rise, thus bringing forward the projected date when the federal debt will reach its current limit. The House passed an amended version of the Housing and Economic Recovery Act of 2008 that included a debt limit increase to \$10,615 billion on July 23, 2008. The Senate passed the measure on July 26, and the President signed it on July 30, raising the debt limit for the first time in 2008. Subsequently, the Emergency Economic Stabilization Act of 2008, signed into law on October 3, raised the debt limit for the second time in 2008 to \$11,315 billion. The debt limit was raised for the third time in less than a year as a result of passage of the American Recovery and Reinvestment Act of 2009. President Obama signed this measure on February 17, 2009, which raised the debt limit to \$12,104 billion.
\nextline
The debt limit was again raised in late 2009. H.R. 4314, passed by the House on December 16, 2009, and by the Senate on December 24, raised the debt limit to \$12,394 billion when the President signed the measure on December 28. In early 2010, Congress voted for a larger increase in the debt limit, which would put the limit at \$14,294 billion, raising the debt ceiling by \$1,900 billion. President Obama signed the measure on February 12, 2010. This debt limit, according to projections, will allow the Treasury Department to issue new debt at projected levels until spring 2011.
\nextline
Amendments offered during consideration of the latest debt limit increase may signal a growing concern with the fiscal sustainability. One amendment, which was not approved, would have established a statutory commission to consider long-term fiscal issues. Another amendment, which was approved, would impose certain PAYGO restrictions. President Obama then created the National Commission on Fiscal Responsibility and Reform (Fiscal Commission), which was charged with identifying “policies to improve the fiscal situation in the medium term and to achieve fiscal sustainability over the long run.” A Fiscal Commission report, along with several other budget policy proposals, could provide one framework for changes that might strengthen the federal government’s fiscal outlook.
\nextline
Over the next decade, without major changes in federal policies, persistent and possibly growing deficits, along with the ongoing growth in the debt holdings of government accounts, would increase substantially the amount of federal debt subject to limit. Unless federal policies change, Congress would repeatedly face demands to raise the debt limit to accommodate the growing federal debt in order to provide the government with the means to meet its financial obligations.