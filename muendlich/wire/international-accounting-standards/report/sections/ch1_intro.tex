\section{Introduction and definition}
\label{sec:IntroductionAndDefinition}

An International Accounting Standard, commonly referred to as IAS, can be viewed as a common global language for business affairs so that company accounts are understandable and comparable across international boundaries. They are a consequence of growing international shareholding and trade and are particularly important for companies that have dealings in several countries. They are rules to be followed by accountants to maintain books of accounts which are comparable, understandable, reliable and relevant. \\

Something that may seem confusing is the difference between the terms \textit{International Accounting Standard (\textbf{IAS})} and \textit{International Financial Reporting Standard (\textbf{IFRS})}. However, in reality, there is no actual difference between these two as the IAS and its publishing organisation, the \textbf{International Accounting Standards Committee (IASC)} have been developed into the IRFS published by the now called \textbf{International Accounting Standards Board (IASB)}. \\

IFRS are used in many parts of the world, including but not limited to the European Union, India, Hong Kong, Australia, Malaysia, Pakistan, GCC countries, Russia, Chile, Philippines, South Africa, Singapore and Turkey. The United States are currently considering the adoption of the IRFS; momentarily they use Generally accepted accounting principles (GAAP), but are working with the IASB together.