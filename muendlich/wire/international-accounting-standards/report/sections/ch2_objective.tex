\section{Objective of financial statements}
\label{Objective}

\subsection{Overview}
IAS 1 (\textit{Presentation of Financial Statements}) sets out the overall requirements for financial statements, including how they should be structured, the minimum requirements for their content and overriding concepts such as going concern, the accrual basis of accounting and the current/non-current distinction. The standard requires a complete set of financial statements to comprise a statement of financial position, a statement of profit or loss and other comprehensive income, a statement of changes in equity and a statement of cash flows. \\

IAS 1 was reissued in September 2007 and applies to annual periods beginning on or after 1 January 2009.

\subsection{Objective of IAS 1}
The objective of IAS 1 (2007) is to prescribe the basis for presentation of general purpose financial statements, to ensure comparability both with the entity's financial statements of previous periods and with the financial statements of other entities. IAS 1 sets out the overall requirements for the presentation of financial statements, guidelines for their structure and minimum requirements for their content. Standards for recognising, measuring, and disclosing specific transactions are addressed in other Standards and Interpretations.

\subsection{Scope}
IAS 1 applies to all general purpose financial statements that are prepared and presented in accordance with International Financial Reporting Standards (IFRSs). \\

General purpose financial statements are those intended to serve users who are not in a position to require financial reports tailored to their particular information needs.

\subsection{Objective of financial statements}
The objective of general purpose financial statements is to provide information about the financial position, financial performance, and cash flows of an entity that is useful to a wide range of users in making economic decisions. To meet that objective, financial statements provide information about an entity's:
\begin{itemize}
	\item assets 
	\item liabilities
	\item equity
	\item income and expenses, including gains and losses
	\item contributions by and distributions to owners (in their capacity as owners) 
	\item cash flows
\end{itemize}

That information, along with other information in the notes, assists users of financial statements in predicting the entity's future cash flows and, in particular, their timing and certainty.

\subsection{Components of financial statements}
A complete set of financial statements includes:
\begin{itemize}
	\item a statement of financial position (balance sheet) at the end of the period 
	\item a statement of profit or loss and other comprehensive income for the period (presented as a single statement, or by presenting the profit or loss section in a separate statement of profit or loss, immediately followed by a statement presenting comprehensive income beginning with profit or loss) 
	\item a statement of changes in equity for the period 
	\item a statement of cash flows for the period notes, comprising a summary of significant accounting policies and other explanatory notes 
	\item comparative information prescribed by the standard.
\end{itemize}

An entity may use titles for the statements other than those stated above.  All financial statements are required to be presented with equal prominence. \\ 

When an entity applies an accounting policy retrospectively or makes a retrospective restatement of items in its financial statements, or when it reclassifies items in its financial statements, it must also present a statement of financial position (balance sheet) as at the beginning of the earliest comparative period. \\

Reports that are presented outside of the financial statements – including financial reviews by management, environmental reports, and value added statements – are outside the scope of IFRSs.