\section{Special legislative procedure}
\label{sec:SLP}

The EU uses a wide variety of special legislative procedures. Such procedures are only used when the Treaties explicitly refer one these procedure is to be used for a particular policy area or subject. \\

\textbf{Agreement procedure}
The consent procedure is one of the special legislative procedures used in the European Union. The consent procedure is used for adopting most international agreements. The word consent refers to the role the European Parliament (EP) and the Council of Ministers (Council) play in the procedure. Both can either approve or disapprove a proposal, but neither can amend it. //

\textbf{Budgetary procedure}
The budgetary procedure is one of the special legislative procedures used in the European Union. It is used to set the annual budget of the European Union. In determining the multiannual financial framework, the assent procedure is used. For granting discharge on the budget a different procedure is used. \\

\textbf{Assent procedure}
This procedure is one of the special legislative procedures used in the European Union. The assent procedure is used for several very important decisions, as well as for matters where the member states wish to retain a larger degree of control. The word assent refers to the role the European Parliament (EP) plays in the procedure. It has to approve or disapprove a proposal, but cannot amend it. \\

\textbf{Discharge procedure}
This procedure is one of the special legislative procedures i that are used within the European Union. The discharge procedure is used to approve the discharge of the budget. \\

\textbf{Open method of coordination}
This procedure is one of the special legislative procedures used in the European Union. The open method of coordination (OMC) is applied to policy areas where member states are in full control, but where they also wish to coordinate their policies on a particular subject. Decisions that are based on the open coordination method are non-binding; member states are not held accountable for whether or not they implement decisions. The procedure is not part of the European treaties.

\clearpage
\textbf{Procedure for amendment of the Treaties}
The European Treaties might be amended by using one of three different procedures. These procedures are rarely used, but are of great importance for the functioning of the European Union and the way decisions are made within the European Union.
\begin{itemize}
	\item ordinary revision procedure 
	\item simplified revision procedures 
	\item passarelle 
\end{itemize}

\textbf{Procedure without participation of the European Parliament}
The European Treaties might be amended by using one of three different procedures. These procedures are rarely used, but are of great importance for the functioning of the European Union and the way decisions are made within the European Union. \\

\textbf{Consultation procedure}
This procedure is one of the special legislative procedures used in the European Union. The consultation procedure is used for politically sensitive issues, where the member states bear responsibility for policy making and where the member states make decisions based on unanimity. \\

The cooperation procedure i is no longer used with the coming into force of the Lisbon Treaty

\clearpage
\subsection{Procedures establishing secondary legislation}

Much of the EU's regulatory work involves secondary legislation. Primary legal acts establish which of the procedures for secondary legislation is to be used. Firm guidelines and rules govern that choice. \\

\textbf{Non-legislative procedures}
Many of the legal acts that are adopted in the European Union are of a general nature. The practical details of these legal acts are dealt with through secondary legislation. Secondary legislation cannot exceed the framework established in the general act. However, specific measures can still yield significant effect. The procedure for adopting secondary legislation depends on which of three procedures is used.
\begin{itemize}
\item Procedure for delegated acts 
\item Examination procedure for implementing acts  
\item Consultation procedure for implementing acts 
\item Lamfalussy procedure 
\end{itemize}

\textbf{Social protocol}
For proposals that relate to social policy issues the European Commission is obliged to involve social partners in the decision making process. The involvement of the social partners may move beyond an advisory role: member states can decide to have social partners draw up secondary legislation, albeit within certain boundaries. This apsect is what makes the social protocol different from the other procedures for establishing secondary legislation. However, the social protocol is only possible in a limited number of social policy areas. 