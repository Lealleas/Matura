\section{Institutions and Organisations}
\label{sec:InstitutionsAndOrganisations}

There are several insitutions and other bodies in the European Union - the ones highligted are those that have are part of the legestlative procedure.

\begin{itemize}
\item \textbf{European Parliament}
\item European Council
\item \textbf{Council of the European Union}
\item \textbf{European Commission}
\item Court of Justice of the European Union (CJEU)
\item European Central Bank (ECB)
\item European Court of Auditors (ECA)
\item European External Action Service (EEAS)
\item European Economic and Social Committee (EESC)
\item Committee of the Regions (CoR)
\item European Investment Bank (EIB)
\item European Ombudsman
\item European Data Protection Supervisor (EDPS)
\item Interinstitutional bodies
\end{itemize}

\subsection{European Parliament}
\begin{description}
	\item[Role] Directly-elected EU body with legislative, supervisory, and budgetary responsibilities
	\item[Members] 751 MEPs (Members of the European Parliament)
	\item[President] Martin Schulz
	\item[Established in] 1952 as Common Assembly of the European Coal and Steel Community, 1962 as European Parliament, first direct elections in 1979
	\item[Location] Strasbourg (France), Brussels (Belgium), Luxembourg
	\item[Website] \href{http://www.europarl.europa.eu/portal/}{http://www.europarl.europa.eu/portal/}
\end{description}

The European Parliament is the EU's \textbf{law-making body}. It is \textbf{directly elected by EU}voters every 5 years. The last elections were in May 2014.
\nextline

\subsubsection{What does the Parliament do?}
The Parliament has 3 main roles:

	\textbf{Legislative}
	\begin{itemize}
		\item Passing EU laws, together with the Council of the EU, based on European Commission proposals
		\item Deciding on international agreements
		\item Deciding on enlargements
		\item Reviewing the Commission's work programme and asking it to propose legislation
	\end{itemize}

	\textbf{Supervisory}
	\begin{itemize}
		\item Democratic scrutiny of all EU institutions
		\item Electing the Commission President and approving the Commission as a body. Possibility of voting a motion of censure, obliging the Commission to resign
		\item Granting discharge, i.e. approving the way EU budgets have been spent
		\item Examining citizens' petitions and setting up inquiries
		\item Discussing monetary policy with the European Central Bank
		\item Questioning Commission and Council
		\item Election observations
	\end{itemize}

	\textbf{Budgetary}
	\begin{itemize}
		\item Establishing the EU budget, together with the Council
		\item Approving the EU's long-term budget, the Multiannual Financial Framework
	\end{itemize}

\subsubsection{Composition}
The number of MEPs for each country is roughly proportionate to its population, but this is by degressive proportionality: no country can have fewer than 6 or more than 96 MEPs and the total number cannot exceed 751 (750 plus the President). MEPs are grouped by political affiliation, not by nationality.
\\
The President represents Parliament to other EU institutions and the outside world and gives the final go-ahead to the EU budget.

\subsubsection{How does the Parliament work?}

Parliament's work comprises two main stages:
\begin{itemize}
	\item Committees - to prepare legislation. \\
	The Parliament numbers 20 committees and two subcommittees, each handling a particular policy area. The committees examine proposals for legislation, and MEPs and political groups can put forward amendments or propose to reject a bill. These issues are also debated within the political groups.
	\item Plenary sessions – to pass legislation. \\
	This is when all the MEPs gather in the chamber to give a final vote on the proposed legislation and the proposed amendments. Normally held in Strasbourg for four days a month, but sometimes there are additional sessions in Brussels.
\end{itemize}


\subsubsection{The Parliament and you}
If you want to ask the Parliament to act on a certain issue, you can petition it (either by post or online).
Petitions can cover any subject which comes under the EU's remit.
To submit a petition, you must be a citizen of an EU member state or be resident in the EU. Companies or other organisations must be based here.
Other ways of getting in touch with Parliament include contacting your local MEP or the European Parliament Information Office in your country.


\subsection{Council of the European Union}
\begin{description}
	\item[Role] Voice of EU member governments, adopting EU laws and coordinating EU policies
	\item[Members] Government ministers from each EU country, according to the policy area to be discussed
	\item[President] Each EU country holds the presidency on a 6-month rotating basis
	\item[Established in] 1958 (as Council of the European Economic Community)
	\item[Location] Brussels (Belgium)
	\item[Website] \href{http://www.consilium.europa.eu/en/home/}{http://www.consilium.europa.eu/en/home/}
\end{description}

In the Council, \textbf{government ministers from each EU country }meet to discuss, amend and adopt laws, and coordinate policies. The ministers have the authority to commit their governments to the actions agreed on in the meetings.
\\
Together with the European Parliament , the Council is the \textbf{main decision-making body} of the EU.

\clearpage
Not to be confused with:
\begin{itemize}
	\item European Council - quarterly summits, where EU leaders meet to set the broad direction of EU policy making
	\item Council of Europe - not an EU body at all.
\end{itemize}

\subsubsection{What does the Council do?}
\begin{itemize}
	\item \textbf{Negotiates and adopts EU laws}, together with the European Parliament, based on proposals from the European Commission
	\item \textbf{Coordinates} EU countries' policies
	\item Develops the EU's \textbf{foreign and security policy}, based on European Council guidelines
	\item Concludes \textbf{agreements} between the EU and other countries or international organisations
	\item Adopts the annual EU budget - jointly with the European Parliament.
\end{itemize}

\subsubsection{Composition}
There are \textbf{no fixed members }of the EU Council. Instead, the Council meets in 10 different configurations, each corresponding to the policy area being discussed. Depending on the configuration, each country sends their minister responsible for that policy area.
\\
For example, when the Council meeting on economic and financial affairs (the Ecofin Council) is held, it is attended by each country's finance minister.

\textbf{Who chairs the meetings?}
The Foreign Affairs Council has a permanent chairperson - the EU High Representative for Foreign Affairs and Security Policy. All other Council meetings are chaired by the relevant minister of the country holding the rotating EU presidency.
\\
For example, any Environment Council meeting in the period when Estonia holds the presidency will be chaired by the Estonian environment minister.
\\
\textbf{Overall consistency }is ensured by the General Affairs Council - which is supported by the Permanent Representatives Committee. This is composed of EU countries' Permanent Representatives to the EU, who are, in effect, national ambassadors to the EU.
\nextline
\textbf{Eurozone countries}
\\
Eurozone countries coordinate their textbf{economic policy} through the Eurogroup, which consists of their economy and finance ministers. It meets the day before Economic and Financial Affairs Council meetings. Agreements reached in Eurogroup gatherings are formally decided upon in the Council the next day, with only ministers of Eurozone countries voting on those issues.
\\

\subsubsection{How does the Council work?}
\begin{itemize}
	\item All \textbf{discussions and votes} take place in public.
	\item To be passed, decisions usually require a qualified majority:
	\begin{itemize}
		\item 55\% of countries (with 28 current members, this means \textbf{16 countries})
		\item representing at least 65\% of total EU population.
	\end{itemize}
\end{itemize}


To \textbf{block a decision}, at least \textbf{4 countries} are needed (representing at least 35\% of total EU population)
\begin{itemize}
\item \textbf{Exception} - sensitive topics like foreign policy and taxation require a unanimous vote (all countries in favour).
\item Simple majority is required for procedural and administrative issues
\end{itemize}

\subsection{European Commission}
\begin{description}
	\item[Role] Promotes the general interest of the EU by proposing and enforcing legislation as well as by implementing policies and the EU budget
	\item[Members] A team or College of Commissioners, 1 from each EU country
	\item[President] Jean-Claude Juncker
	\item[Established in] 1958
	\item[Location] Brussels (Belgium)
	\item[Website] \href{http://www.consilium.europa.eu/en/home/}{http://www.consilium.europa.eu/en/home/}
\end{description}

The European Commission is the EU's \textbf{politically independent executive arm.} It is alone responsible for drawing up proposals for new European legislation, and it implements the decisions of the European Parliament and the Council of the EU.

\clearpage
\subsubsection{What does the Commission do?}

	\textbf{Proposes new laws}
	The Commission is the sole EU institution tabling laws for adoption by the Parliament and the Council that:
	\begin{itemize}
		\item Pprotect the interests of the EU and its citizens on issues that can't be dealt with effectively at national level;
		\item get technical details right by consulting experts and the public.
	\end{itemize}

	\textbf{Manages EU policies and allocates EU funding}
	\begin{itemize}
		\item Sets EU spending priorities, together with the Council and Parliament.
		\item Draws up annual budgets for approval by the Parliament and Council.
		\item Supervises how the money is spent, under scrutiny by the Court of Auditors.
	\end{itemize}

	\textbf{Enforces EU law}
	\begin{itemize}
		\item Together with the Court of Justice, ensures that EU law is properly applied in all the member countries.
	\end{itemize}

	\textbf{Represents the EU internationally}
	\begin{itemize}
		\item peaks on behalf of all EU countries in international bodies, in particular in areas of trade policy and humanitarian aid.
		\item Negotiates international agreements for the EU.
	\end{itemize}

\subsubsection{Composition}
\textbf{Political leadership} is provided by a team of 28 Commissioners (one from each EU country) – led by the Commission President, who decides who is responsible for which policy area.
\\

The College of Commissioners, includes the President of the Commission, his seven Vice-Presidents, including the First Vice-President, and the High-Representative of the Union for Foreign Policy and Security Policy and 20 Commissioners in charge of portfolios.

The \textbf{day-to-day running} of Commission business is performed by its staff (lawyers, economists, etc.), organised into departments known as Directorates-General (DGs), each responsible for a specific policy area. 

\clearpage
\textbf{Appointing the President}

The candidate is put forward by national leaders in the European Council, taking account of the results of the European Parliament elections. He or she needs the support of a majority of members of the European Parliament in order to be elected.

\textbf{Selecting the team}

The Presidential candidate selects potential Vice-Presidents and Commissioners based on suggestions from the EU countries. The list of nominees has to be approved by national leaders in the European Council.
Each nominee appears before the European Parliament to explain their vision and answer questions. Parliament then votes on whether to accept the nominees as a team. Finally, they are appointed by the European Council, by a qualified majority.
The current Commission's term of office runs until 31 October 2019.

\subsubsection{How does the Commission work?}
\textbf{Strategic planning}
The President defines the policy direction for the Commission, which enables the Commissioners together to decide strategic objectives, and produce the annual work programme.

\textbf{Collective decision making}
Decisions are taken based on collective responsibility. All Commissioners are equal in the decision-making process and equally accountable for these decisions. They do not have any individual decision-making powers, except when authorized in certain situations.
\\

The Vice-Presidents act on behalf of the President and coordinate work in their area of responsibility, together with several Commissioners. Priority projects are defined to help ensure that the College works together in a close and flexible manner.
\\

Commissioners support Vice-Presidents in submitting proposals to the College. In general, decisions are made by consensus, but votes can also take place. In this case, decisions are taken by simple majority, where every Commissioner has one vote.
\\

The relevant Directorate-General (headed by a Director-General, answerable to the relevant Commissioner) then takes up the subject. This usually done in the form of draft legislative proposals.
These are then resubmitted to the Commissioners for adoption at their weekly meeting, after which they become official, and are sent to the Council and the Parliament for the next stage in the EU legislative process.
\clearpage














