\section{The Debt Ceiling in the last decade}
\label{sec:LastDecade}

\subsection{The Dept Limit Issue in 2002}
Accumulating debt in government accounts produced most of the pressure on the debt limit that occurred early in 2002. As deficits reemerged in 2002, increases in debt held by the public added to the pressure on the debt limit in the spring of 2002. During the four fiscal years with surpluses (1998-2001), the increases in federally held debt and decreases in debt held by the public produced a net increase of \$405 billion in total debt subject to limit. At the beginning of 2002 (October 1, 2001), debt subject to limit was within \$217 billion of the existing \$5,950 billion debt limit. Between then and the end of May 2002, debt subject to limit increased by another \$217 billion, divided between a \$117 billion increase in debt held by government accounts and a \$100 billion increase in debt held by the public, putting the debt close to the \$5,950 billion limit. 
\newline \newline
In the fall of 2001, the Administration recognized that a deteriorating budget outlook and continued growth in debt held by government accounts were likely to lead to the debt limit soon being reached. In early December 2001, it asked Congress to raise the debt limit by \$750 billion to \$6,700 billion. As the debt moved closer to and reached the debt limit over the first six months of 2002, the Administration asked Congress repeatedly to increase the debt limit, warning of adverse financial consequences were the limit not raised.
\newline \newline
On April 4, 2002, the Treasury held debt below the limit by invoking its legislatively mandated authority to suspend reinvestment of government securities in the G-Fund of the federal employees’ Thrift Savings Plan (TSP). This allowed the Treasury to issue new debt and meet the government’s obligations. On April 15, debt subject to limit stood at \$5,949,975 million, just \$25 million below the limit. Once April 15 tax revenues flowed in, the Treasury “made whole” the G- Fund by restoring all of the debt that had not been issued to the TSP over this period and crediting the fund with interest it would have earned on that debt. By the end of April, debt subject to limit had fallen back \$35 billion below the limit.
\newline \newline

\textbf{Resolving the Debt Limit Issue in 2002}
\newline
By the middle of May 2002, debt subject to limit had again risen to within \$15 million of the statutory limit. At the 2002 average spending rate, \$15 million equaled about five minutes of federal outlays. The Treasury, for the second time in 2002, used its statutory authority to avoid a default. The Treasury’s financing problems, however, would persist without an increase in the debt limit. On May 14, the Treasury asked Congress to raise the debt limit or enact other statutory changes allowing the Treasury to issue new debt. A Treasury news release stated “absent extraordinary actions, the government will exceed the statutory debt ceiling no later than May 16,” and that a “debt issuance suspension period” will begin no later than May 16 [2002].... [This] allows the Treasury to suspend or redeem investments in two trust funds, which will provide flexibility to fund the operations of the government during this period.
\newline \newline
The Treasury reduced federal debt held by these government accounts by replacing it with non- interest-bearing, non-debt instruments, which enabled it to issue new debt to meet the government’s obligations. The Treasury claimed these extraordinary actions would suffice, at the latest, through June 28, 2002. Without a debt limit increase by that date, the Treasury indicated it would need to take other actions to avoid breaching the ceiling. By June 21, the Treasury had postponed a regular securities auction, but took no other actions. With large payments and other obligations due at the end of June and at the beginning of July, the Treasury stated it would soon exhaust all options to issue debt and fulfill government obligations, putting the government on the verge of a default.
During May and June 2002, Congress took steps to increase the debt limit. The 2002 supplemental appropriations bill passed by the House on May 24 included, after extended debate, language allowing any eventual House-Senate conference on the legislation to increase the debt limit. However, the Senate’s supplemental appropriations bill (June 3, 2002) omitted debt-limit-increasing language. The Senate leadership expressed strong reluctance to include a debt limit increase in the supplemental appropriation bill. Instead, on June 11, the Senate adopted a bill, without debate, to raise the debt limit by \$450 billion to \$6,400 billion. At that time, a \$450 billion debt limit increase was thought to provide enough borrowing authority for government operations through the rest of calendar year 2002, if not through the summer of 2003. With the possibility of default looming over it, the House passed the \$450 billion debt limit increase by a single vote on June 27. The President signed the bill into law on June 28, ending the 2002 debt limit crisis.


\subsection{The Dept Limit Issue in 2005, 2006, and 2007}
Debt limit increases in 2005, 2006, and 2007 took a less dramatic path than those in President Bush’s first term. In 2005, Congress included three reconciliation instructions in the 2006 budget resolution (April 28, 2005), the third of which directed the House Committee on Ways and Means and the Senate Finance Committee to report bills raising the debt limit. The instructions specified a \$781 billion debt limit increase, to \$8,965 billion, with a reporting date of no later than September 30, 2005. Neither committee reported a bill to raise the debt limit.
\newline \newline
The adoption of the conference report on the 2006 budget resolution in late April 2005 also triggered the Gephardt rule (House Rule XXVII), producing a House Joint Resolution that also would raise the debt limit by \$781 billion to \$8,965 billion. Under the rule, the resolution was automatically deemed passed by the House and sent to the Senate. Through the end of the first session of the 109th Congress, the Senate had not considered H.J.Res., nor had Congress considered a reconciliation bill raising the debt limit as called for in the budget resolution.
\newline \newline
At the end of December 2005, Secretary of the Treasury Snow wrote Congress that the debt limit would probably be reached in mid-February 2006, although the Treasury could take actions that maintain the debt below its limit until mid-March. He therefore requested an increase in the debt limit. In two more letters, sent on February 19 and March 6, Secretary Snow advised Congress that the Treasury was taking measures within its legal discretion to avoid reaching the limit and that these measures would suffice only until the middle of March 2006. Secretary Snow authorized actions used previously by the Treasury, including declaring a debt issuance suspension period. As March began, the government was again close to becoming unable to meet its obligations. During the week of March 13 the Senate took up H.J.Res. On March 16, the Senate passed a debt limit increase after rejecting several amendments. The President’s signature on March 20, 2006, then raised the debt limit to \$8,965 billion.
\newline \newline
In mid-May 2007, Congress passed the conference report on the 2008 budget resolution. The House’s Gephardt rule, triggered by the adoption of the conference report on the budget resolution, resulted in the automatic engrossment of a joint resolution raising the debt limit by \$850 billion to \$9,815 billion, and sending it to the Senate. At the end of July 2007, the Treasury asked Congress to raise the debt limit, stating the limit would be reached in early October 2007. In August, the CBO Director said that projections suggested that the limit would be reached in late October or early November. Without an increase, the Treasury indicated that it would take steps within its legal authority to avoid exceeding the debt limit. The Senate Finance Committee approved the House resolution without changes on September 12, 2007. The Senate then passed the measure on September 27, which the President signed on September 29, 2007.

\subsection{The Dept Limit Issue in 2013, 2014, and 2015}
On December 26, 2012, Secretary Geithner sent a letter to Congress stating that the debt limit, the last increase provided for under the BCA, would be reached on December 31, 2012. Treasury estimated that the use of extraordinary measures would provide additional headroom under the debt limit until early 2013. A debt issuance suspension period was declared on December 31, 2012, at which time Treasury prematurely redeemed securities of the Civil Service Retirement and Disability Trust Fund and did not invest receipts of the Civil Service Retirement and Disability Trust Fund and the Postal Service Retiree Health Benefit Fund. On January 15, 2013, Secretary Geithner notified Congress that Treasury would suspend investments in the Government Securities Investment Fund (G-Fund) of the Federal Thrift Savings Plan. On February 4, 2013, the statutory debt limit was suspended through May 18, 2013, as part of the No Budget, No Pay Act of 2013.
\newline
On May 19, 2013, the debt limit was reinstated and raised to \$16,699 billion, a level which accommodated borrowing incurred during the suspension period. The issuance of SLGS Treasury securities was suspended until further notice on May 15, 2013. A debt issuance suspension period was declared on May 20, 2013, at which time Treasury prematurely redeemed securities of the Civil Service Retirement and Disability Trust Fund and did not invest receipts of the Civil Service Retirement and Disability Trust Fund and the Postal Service Retiree Health Benefit Fund. On May 31, 2013, Secretary Lew notified Congress that Treasury would suspend investments in the Government Securities Investment Fund (G-Fund) of the Federal Thrift
\newline \newline
On February 7, 2014, Secretary Lew sent a letter to Congress stating that at the end of the debt limit suspension period, Treasury would begin utilizing the extraordinary measures to finance government operations. Treasury estimated that the use of these extraordinary measures would provide additional headroom under the debt limit until February 27, 2014. On February 10, 2014 (the next business day after the suspension period ended), the debt limit was reinstated and raised to \$17,212 billion, a level which accommodated borrowing incurred during the suspension period. A debt issuance suspension period was declared on February 10, 2014, at which time Treasury prematurely redeemed securities of the Civil Service Retirement and Disability Trust Fund and suspended investments in the Government Securities Investment Fund (G-Fund) of the Federal Thrift Savings Plan. On February 15, 2014, the debt limit was suspended for a third time through March 15, 2015, as part of the Temporary Debt Limit Extension Act.
\newline \newline
On March 13, 2015, Secretary Lew sent a letter to Congress stating that at the end of the debt limit suspension period, Treasury would begin using the extraordinary measures to finance government operations.33 On March 16, 2015, the debt limit was reinstated and raised to \$18,113 billion, a level which accommodated borrowing incurred during the suspension period. A debt issuance suspension period was declared on March 16, 2015, at which time Treasury prematurely redeemed securities and suspended investment of the Civil Service Retirement and Disability Trust Fund and Postal Service Retiree Health Benefits Fund and suspended investments in the Government Securities Investment Fund (G-Fund) of the Federal Thrift Savings Plan.34 The Congressional Budget Office recently estimated the extraordinary measures would be exhausted around October or November 2015.
