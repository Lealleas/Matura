\section{Introduction}
\label{sec:Introduction}

The debt limit is the total amount of money that the United States government is authorized to borrow to meet its existing legal obligations, including Social Security and Medicare benefits, military salaries, interest on the national debt, tax refunds, and other payments. The debt limit does not authorize new spending commitments. It simply allows the government to finance existing legal obligations that Congresses and presidents of both parties have made in the past.
\newline \newline 
Failing to increase the debt limit would have catastrophic economic consequences. It would cause the government to default on its legal obligations – an unprecedented event in American history. That would precipitate another financial crisis and threaten the jobs and savings of everyday Americans – putting the United States right back in a deep economic hole, just as the country is recovering from the recent recession.  
\newline \newline
Congress has always acted when called upon to raise the debt limit. Since 1960, Congress has acted 78 separate times to permanently raise, temporarily extend, or revise the definition of the debt limit – 49 times under Republican presidents and 29 times under Democratic presidents.  In the coming weeks, Congress must act to increase the debt limit. Congressional leaders in both parties have recognized that this is necessary. Recently, however, a number of myths about this issue have begun to surface.

\subsection{Federal Government Debt and the Debt Limit}
The gross federal debt, which represents the federal government’s total outstanding debt, consists of
\begin{itemize}
\item the debt held by the public and
\item the debt held in government accounts, also known as intragovernmental debt.
\end{itemize}
Federal government borrowing increases for two primary reasons: (1) budget deficits and
(2) investments of any federal government account surpluses in Treasury securities as required by law. \footnote{If the budget is in surplus and intragovernmental debt rises by an amount that is less than the budget surplus, the total debt would not increase. See the later discussion in the section titled “Implications of Future Federal Debt on the
Debt Limit.”}
\newline \newline
The debt held by the public represents the total net amount borrowed from the public to cover the federal government’s accumulated budget deficits. Annual budget deficits increase the debt held by the public by requiring the federal government to borrow additional funds to fulfill its commitments.
\newline \newline
The debt held in government accounts represents the federal debt issued to certain accounts, primarily trust funds, such as those associated with Social Security, Medicare, and Unemployment Compensation. Generally, government account surpluses, which include trust fund surpluses, by law must be invested in special non-marketable federal government securities and thus are held in the form of federal debt. Treasury periodically pays interest on the special securities held in a government account. Interest payments are typically paid in the form of additional special securities issued by Treasury to the trust funds, which also increases the amount of intragovernmental debt and federal debt subject to limit.
\newline \newline
When a trust fund invests in U.S. Treasury securities, it effectively lends money to the rest of the government. The loan either reduces what the federal government must borrow from the public if the budget is in deficit, or reduces the amount of publicly held debt if the budget is in surplus. At the same time, the loan increases intragovernmental debt. The revenues exchanged for these securities then go into the General Fund of the Treasury and are indistinguishable from other cash in the General Fund. This cash may be used for any government spending purpose.
\newline \newline
Congress created a statutory debt limit in the Second Liberty Bond Act of 1917.7 This development changed Treasury’s borrowing process and assisted Congress in its efforts to exercise its constitutional prerogatives to control the federal government’s fiscal outcomes. The debt limit also imposes a form of fiscal accountability that compels Congress and the President to take deliberate action to allow further federal borrowing if necessary.
\newline \newline
Almost all of the federal government’s borrowing is subject to a statutory limit.8 From time to time, Congress has considered and adopted legislation to change or suspend this limit. Because the statutory limit applies to debt held by the public as well as intragovernmental debt, both budget deficits and government account surpluses may contribute to the federal government reaching the existing debt limit.
\subsection{The Debt Limit and the Treasury}
Treasury’s standard methods for financing federal activities can be disrupted when the level of federal debt nears its legal limit. If the limit prevents Treasury from issuing new debt to manage short-term cash flows or to finance an annual deficit, the government may be unable to obtain the cash needed to pay its bills. The limit may also prevent the government from issuing new debt in order to invest the surpluses of designated government accounts, such as federal trust funds. Treasury is caught between two requirements: the law that requires Treasury to pay the government’s legal obligations or invest trust fund surpluses, on one hand, and the statutory debt limit which may prevent Treasury from issuing the debt to raise cash to pay obligations or make trust fund investments, on the other.
\newline \newline
The level of federal debt changes throughout the year due to fluctuations in revenue and outlays, regardless of whether or not the government has an annual surplus or deficit. Seasonal fluctuations could still require Treasury to sell debt even if the annual level of federal debt subject to limit does not increase (i.e., if the budget were balanced and trust funds were not in surplus). Even on a day-to-day basis, the level of federal debt can vary significantly. For example, Treasury issues large volumes of individual income tax refunds in February and March, because taxpayers expecting refunds tend to file early. On the other hand, Treasury tends to collect more revenue in April because taxpayers making payments tend to file closer to April 15.
\newline \newline
Past Treasury Secretaries, when faced with a nearly binding debt ceiling, have used special strategies to handle cash and debt management responsibilities.
\newline \newline
Since 1985, these measures have included
\begin{itemize}
\item suspending sales of nonmarketable debt (savings bonds, state and local government series, and other nonmarketable debt);
\item trimming or delaying auctions of marketable securities;
\item under-investing or disinvesting certain government funds (Social Security, Government Securities Investment Fund of the Federal Thrift Savings Plan, the Civil Service Retirement and Disability Trust Fund, Postal Service Retiree Health Benefit Fund, Exchange Stabilization Fund);11 and
\item exchanging Treasury securities for non-Treasury securities held by the Federal Financing Bank (FFB).
\end{itemize}

\subsection{Reasioning behind the Debt Limit}
The debt limit can hinder the Treasury’s ability to manage the federal government’s finances, as noted above. In extreme cases, when the federal debt is very near its statutory limit, the Treasury must take unusual and extraordinary measures to meet federal obligations. While the debt limit has never caused the federal government to default on its obligations, it has at times caused great inconvenience and has added uncertainty to Treasury operations.
\newline \newline
The debt limit also provides Congress with the strings to control the federal purse, allowing Congress to assert its constitutional prerogatives to control spending. The debt limit also imposes a form of fiscal accountability that compels Congress and the President to take visible action to allow further federal borrowing when the federal government spends more than it collects in revenues. In the words of one author, the debt limit “expresses a national devotion to the idea of thrift and to economical management of the fiscal affairs of the government.” On the other hand, some budget experts have advocated elimination of the debt limit, arguing that other controls provided by the modern congressional budget process established in 1974 have superseded the debt limit, and that the limit does little to alter spending and revenue policies that determine the size of the federal deficit.